\documentclass{article}
\usepackage[utf8]{inputenc}
\usepackage{natbib}
\usepackage{graphicx}
\usepackage{caption}

\usepackage[margin=1.0in]{geometry}

\pagestyle{myheadings}
\markright{\hfill CALIFORNIA POLYTECHNIC STATE UNIVERSITY, SAN LUIS OBISPO\hfill}

\title{Binary Packers(Cryptors)}
\author{Murtaza Munaim}
\date{June 15 2013}

\begin{document}
\maketitle

\section{Abstract}
The compiler  a vital component  piece of source code, parse the syntactic structure of the text, apply optimizations, and produce opcodes for the
target machine that the code is supposed to run on be it x86, ARM, MIPS, etc.  take a piece of source code, do some analysis, and produce and executable file. 
We configured and measured the output and input voltages of various
operational amplifier circuit configurations.

\section{Procedure}
\begin{enumerate}
\item Built an inverting amplifier using a static $R_{in} = 10 k\Omega $ \newline and $R_{f} = [10k\Omega, 15k\Omega, 27k\Omega, 47k\Omega, 100k\Omega$] \newline
\includegraphics[width=6cm,height=6cm,keepaspectratio]{invert.jpg}
\item Measured voltage and phase across the $V_{out}$ and $V_{in}$ to calculate actual gain and phase differences for the configurations
\item Increase input voltage until clipping was visible on the op-amp configuration.
\item Repeated steps 1-3 but using a non-inverting amplifier configuration \newline 
\includegraphics[width=6cm,height=6cm,keepaspectratio]{noninvert.jpg}
\item Built a voltage follower circuit and applied a sinusoidal input \newline 
\includegraphics[width=6cm,height=6cm,keepaspectratio]{follower.jpg}
\begin{enumerate}
\item Changed input to a $10V_{pp}$ square wave. Increased voltage until clipping was visible.
\item Calculated the slope of the rising edge of the output of the square wave
\end{enumerate}
\item Built a summing amplifier circuit using resistor values of  {$R_{1in} = 10k\Omega$, $R_{2in} = 27k\Omega$, $R_{3in} = \infty$ \newline
$R_{f} = 10k\Omega$} The voltages set to : {$V_{1in} = 2.0V$, $V_{2in} = 2.0V$, $V_{3in} = 0V$}. The output voltage was measured.
\item Built an integrator amplifier circuit using an input resistor of $R_{in} = 100k\Omega$, $V_{in} = 5.0V$, and $C_{f} = 0.05$$\mu$F
\begin{enumerate}
\item Applied a $10V_{pp}$ 500Hz square wave to the input of the integrator
\item Connected a large $1M\Omega$ resistor across the feedback capacitor to correct the signal from drift
\item Measured both the input and output signal of the integrator with 0DC input and AC coupled for clean signal
\end {enumerate}
\end{enumerate}

\section{Experimental Data}

\subsection{Inverting Amplifier}

\begin{table}[!h]\LARGE
    \begin{center}
        \begin{tabular}{|c||c|c|c|} \hline
            $R_{B} (k\Omega)$ & $V_{O}$ (V) & $A_{TH}$ & $A_{EXP}$\\ \hline
            10 & 0.248 & -1 & -1.046 \\ \hline
            15 & 0.344 & -1.5 & -1.45\\ \hline
            27 & 0.575 & -2.7 & -2.43\\ \hline
            47 & 1.000 & -4.7 & -4.35\\ \hline
            100 & 2.06 & -10.0 & -8.72\\ \hline
        \end{tabular}
        \caption{Inverting Amplifier Measurements}
        \label{invertchart}
    \end{center}
\end{table}
\vspace{10pt}

\begin{center}
\includegraphics[width=10cm,height=8cm,keepaspectratio]{1_10k.jpg}
\captionof{figure}{Inverting Amplifier Vi = 0.2V, Rb = 10k}\label{10kinvert}
\end{center}
\vspace{10pt}

\begin{center}
\includegraphics[width=10cm,height=8cm,keepaspectratio]{1_15k.jpg}
\captionof{figure}{Inverting Amplifier Vi = 0.2V, Rb = 15k}\label{15kinvert}
\end{center}
\vspace{10pt}

\begin{center}
\includegraphics[width=10cm,height=8cm,keepaspectratio]{1_27k.jpg}
\captionof{figure}{Inverting Amplifier Vi = 0.2V, Rb = 27k}\label{27kinvert}
\end{center}
\vspace{10pt}

\begin{center}
\includegraphics[width=10cm,height=8cm,keepaspectratio]{1_47k.jpg}
\captionof{figure}{Inverting Amplifier Vi = 0.2V, Rb = 47k}\label{47kinvert}
\end{center}
\vspace{10pt}

\begin{center}
\includegraphics[width=10cm,height=8cm,keepaspectratio]{1_100k.jpg}
\captionof{figure}{Inverting Amplifier Vi = 0.2V, Rb = 100k}\label{100kinvert}
\end{center}
\vspace{10pt}

\begin{center}
\includegraphics[width=10cm,height=8cm,keepaspectratio]{3_sin200.jpg}
\captionof{figure}{10k Inverting Amplifier Provided with 200Hz Input}\label{200invert}
\end{center}
\vspace{10pt}


\subsection{Non-Inverting Amplifier}

\begin{table}[!h]
    \begin{center}
        \begin{tabular}{|c||c|c|c|} \hline
            $R_{B} (k\Omega)$ & $V_{O}$ (V) & $A_{TH}$ & $A_{EXP}$\\ \hline
            10 & 0.356 & 2 & 1.78 \\ \hline
            15 & 0.425 & 2.5 & 2.12\\ \hline
            27 & 0.588 & 3.7 & 2.94\\ \hline
            47 & 0.940 & 5.7 & 4.7\\ \hline
            100 & 1.670 & 11.0 & 8.35\\ \hline
        \end{tabular}
        \caption{Non-Inverting Amplifier Measurements}
        \label{noninvertchart}
    \end{center}
\end{table}
\vspace{10pt}


\begin{center}
\includegraphics[width=10cm,height=8cm,keepaspectratio]{2_10k.jpg}
\captionof{figure}{Non-Inverting Amplifier Vi = 0.2V, Rb = 10k}\label{10knoninvert}
\end{center}
\vspace{10pt}

\begin{center}
\includegraphics[width=10cm,height=8cm,keepaspectratio]{2_15k.jpg}
\captionof{figure}{Non-Inverting Amplifier Vi = 0.2V, Rb = 15k}\label{15knoninvert}
\end{center}
\vspace{10pt}

\begin{center}
\includegraphics[width=10cm,height=8cm,keepaspectratio]{2_27k.jpg}
\captionof{figure}{Non-Inverting Amplifier Vi = 0.2V, Rb = 27k}\label{27knoninvert}
\end{center}
\vspace{10pt}

\begin{center}
\includegraphics[width=10cm,height=8cm,keepaspectratio]{2_47k.jpg}
\captionof{figure}{Non-Inverting Amplifier Vi = 0.2V, Rb = 47k}\label{47knoninvert}
\end{center}
\vspace{10pt}

\begin{center}
\includegraphics[width=10cm,height=8cm,keepaspectratio]{2_100k.jpg}
\captionof{figure}{Non-Inverting Amplifier Vi = 0.2V, Rb = 100k}\label{100knoninvert}
\end{center}
\vspace{10pt}


\subsection{Voltage Follower}

\begin{center}
\includegraphics[width=10cm,height=8cm,keepaspectratio]{clipping_of_opamp.jpg}
\captionof{figure}{Vo Distortion, Rb = 27k}\label{clip27k}
\end{center}
\vspace{10pt}

\begin{center}
\includegraphics[width=10cm,height=8cm,keepaspectratio]{slew_rate.jpg}
\captionof{figure}{Voltage Follower Slew Rate, Vi = 10V}\label{followslew}
\end{center}
\vspace{10pt}

\subsection{Summing Amplifier}

\begin{center}
\includegraphics[width=10cm,height=8cm,keepaspectratio]{summing_4.jpg}
\captionof{figure}{Summing Amplifier}\label{summing}
\end{center}
\vspace{10pt}

\subsection{Integrating Amplifier}

\begin{center}
\includegraphics[width=10cm,height=8cm,keepaspectratio]{integrator.jpg}
\captionof{figure}{Integrator Amplifier}\label{integamp}
\end{center}
\vspace{10pt}


\section{Discussion}
\begin{enumerate}
\item One
\item Two
\item Three
\item Four
\item Five
\end{enumerate}

\section{Conclusions}
The op-amp is awesome.
\subsection{Murtaza Munaim}
Our various builds of op-amps showed some interesting features of the op-amps. The slew-factor was measured and recorded for
our LM741. This is an interesting feature of the op-amp as it describes how tight of a bandwith that this op-amp can be used 
for use in filter design. Another interesting feature of the op-amp I inferred from the data is that as the feedback resistor's
value increased, the op-amps gain's error rate increased as well, deviating more and more from the theoretical gain value.
\subsection{Brandon Ivy}
\subsection{Nathik Azad}

%\bibliographystyle{plain}
%\bibliography{references}
\end{document}
